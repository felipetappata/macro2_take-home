\documentclass[../../entrega.tex]{subfiles}
\begin{document}
\subsection{Inciso (b)}
A lo largo de este inciso, a medida que derivamos las ecuaciones log-linealizadas, mencionamos la referencia que le corresponde en el listado final de las ecuaciones, que se encuentra en la página \pageref{eq:b-1}.
% ==== Begin B.1 ==== %
Empezamos con la ecuación (\ref{eq:2-5}), del problema del padre, que es
\begin{equation*}
    \psi L_t^\chi = C_t^{-\sigma} w_t.
\end{equation*}
Tomando logaritmo de ambos lados, tenemos
\begin{equation*}
    \ln \psi + \chi \ln L_t = -\sigma \ln C_t + \ln w_t.
\end{equation*}
La correspondiente ecuación en estado estacionario es
\begin{equation*}
    \ln \psi + \chi \ln L = -\sigma \ln C + \ln w.
\end{equation*}
Restando la ecuación en estado estacionario de la ecuación ``dinámica'' (i.e., que depende del tiempo), llegamos a
\begin{equation*}
    \chi(\ln L_t - \ln L) = -\sigma(\ln C_t - \ln C) + (\ln w_t - \ln w).
\end{equation*}
La ecuación log-linealizada es por lo tanto
\begin{equation*}
    \chi l_t = -\sigma c_t + \hat{w}_t,
\end{equation*}
donde, siguiendo la notación de los autores, las variables minúsculas sin ``\emph{hat}'' denotan desviaciones logaritmo del estado estacionario de variables mayúsculas, y las variables con ``\emph{hat}'' denotan desviaciones logarítmicas de variables que ya eran minúsculas en su versión nivel.
Esta es la ecuación (\ref{eq:b-1}).

% ==== End B.1 ==== %
% ==== Begin B.2 ==== %
Continuamos con la ecuación (\ref{eq:2-6}), que define el factor de descuento estocástico del padre:
\begin{equation*}
    \Lambda_{t-1,t} = \beta \left(\frac{C_t}{C_{t-1}}\right)^{-\sigma}.
\end{equation*}
Tomando logaritmo de ambos lados:
\begin{equation*}
    \ln \Lambda_{t-1,t} = \ln \beta - \sigma(\ln C_t - \ln C_{t-1}).
\end{equation*}
En estado estacionario, cuando $C_t = C_{t-1} = C$, tenemos
\begin{equation*}
    \ln \Lambda = \ln \beta.
\end{equation*}
Restando la ecuación en estado estacionario:
\begin{equation*}
    \ln \Lambda_{t-1,t} - \ln \Lambda = -\sigma(\ln C_t - \ln C_{t-1}).
\end{equation*}
Como $\lambda_{t-1,t}$ denota la desviación logarítmica de $\Lambda_{t-1,t}$ respecto de su estado estacionario, y $c_t$ denota la desviación logarítmica de $C_t$ respecto de su estado estacionario, tenemos
\begin{equation*}
    \lambda_{t-1,t} = -\sigma(c_t - c_{t-1}).
\end{equation*}
Esta es la ecuación (\ref{eq:b-2}).
% ==== End B.2 ==== % 
% ==== Begin B.3 ==== %
Ahora consideramos la ecuación (\ref{eq:2-7}), que es la ecuación de Euler para bonos de corto plazo del hogar padre:
\begin{equation*}
    1 = R_t^s \mathbb{E}_t \Lambda_{t,t+1} \Pi_{t+1}^{-1}.
\end{equation*}
Tomando logaritmo de los dos lados, obtenemos
\begin{equation*}
    0 = \ln R_t^s + \ln [\mathbb{E}_t \Lambda_{t,t+1} \Pi_{t+1}^{-1}],
\end{equation*}
que en estado estacionario es
\begin{equation*}
    0 = \ln R^s + \ln [\Lambda \Pi^{-1}].
\end{equation*}
Restando la ecuación en estado estacionario de la corriente, tenemos
\begin{equation*}
    0 = (\ln R_t^s - \ln R^s) + (\ln [\mathbb{E}_t \Lambda_{t,t+1} \Pi_{t+1}^{-1}] - \ln [\Lambda \Pi^{-1}]),
\end{equation*}
y usando las definiciones de las variables en desviación logarítmica llegamos a
\begin{equation*}
    0 = r_t^s + \mathbb{E}_t \lambda_{t,t+1} - \mathbb{E}_t \pi_{t+1},
\end{equation*}
que es la ecuación (\ref{eq:b-3}).
% ==== End B.3 ==== %
% ==== Begin B.4 ==== %
La ecuación (\ref{eq:2-13}) define el factor de descuento estocástico del hijo,
\begin{equation*}
    \Lambda_{b,t-1,t} = \beta_b \left(\frac{C_{b,t}}{C_{b,t-1}}\right)^{-\sigma},
\end{equation*}
y tomando logaritmo de ambos lados obtenemos
\begin{equation*}
    \ln \Lambda_{b,t-1,t} = \ln \beta_b - \sigma(\ln C_{b,t} - \ln C_{b,t-1}).
\end{equation*}
En estado estacionario, cuando $C_{b,t} = C_{b,t-1} = C_b$, tenemos
\begin{equation*}
    \ln \Lambda_b = \ln \beta_b,
\end{equation*}
y restando la ecuación en estado estacionario llegamos a
\begin{equation*}
    \ln \Lambda_{b,t-1,t} - \ln \Lambda_b = -\sigma(\ln C_{b,t} - \ln C_{b,t-1}),
\end{equation*}
que en términos de desviaciones logarítmicas es
\begin{equation*}
    \lambda_{b,t-1,t} = -\sigma(c_{b,t} - c_{b,t-1}),
\end{equation*}
correspondiente a la ecuación (\ref{eq:b-4}).
% ==== End B.4 ==== %
% ==== Begin B.5 ==== %
La ecuación (\ref{eq:2-12}), que define el retorno del bono largo, es
\begin{equation*}
    R_t^b = \frac{1 + \kappa Q_t}{Q_{t-1}}.
\end{equation*}
Tomando logaritmo,
\begin{equation*}
    \ln R_t^b = \ln(1 + \kappa Q_t) - \ln Q_{t-1}.
\end{equation*}
En estado estacionario,
\begin{equation*}
    \ln R^b = \ln(1 + \kappa Q) - \ln Q.
\end{equation*}
Restando la ecuación en estado estacionario,
\begin{equation*}
    \ln R_t^b - \ln R^b = [\ln(1 + \kappa Q_t) - \ln(1 + \kappa Q)] - (\ln Q_{t-1} - \ln Q).
\end{equation*}
Para linealizar el término $\ln(1 + \kappa Q_t)$, usamos la aproximación de Taylor de primer orden de $\ln f(x)$ alrededor de $\bar{x}$:
\begin{equation*}
    \ln f(x) \approx \ln f(\bar{x}) + \frac{f'(\bar{x})}{f(\bar{x})}(x - \bar{x}).
\end{equation*}
En nuestro caso, $f(Q_t) = 1 + \kappa Q_t$ y $\bar{x} = Q$, por lo que:
\begin{equation*}
    \ln(1 + \kappa Q_t) \approx \ln(1 + \kappa Q) + \frac{\kappa}{1 + \kappa Q}(Q_t - Q).
\end{equation*}
Además, notemos que en estado estacionario:
\begin{equation*}
    R^b = \frac{1 + \kappa Q}{Q} \implies 1 + \kappa Q = R^b Q.
\end{equation*}
Por lo tanto:
\begin{equation*}
    \ln R_t^b - \ln R^b = \frac{\kappa}{R^b Q}(Q_t - Q) - (\ln Q_{t-1} - \ln Q),
\end{equation*}
que en términos de desviaciones logarítmicas es
\begin{equation*}
    r_t^b = \frac{\kappa}{R^b}q_t - q_{t-1},
\end{equation*}
que es la ecuación (\ref{eq:b-5}).
% ==== End B.5 ==== %

Resumiendo, la lista completa de condiciones de equilibrio linealizadas es
\begin{gather}
    \chi l_{t}=-\sigma c_{t}+\widehat{w}_{t} \label{eq:b-1} \tag{B.1}\\
    \lambda_{t-1, t}=-\sigma\left(c_{t}-c_{t-1}\right) \label{eq:b-2} \tag{B.2}\\
    0=\mathbb{E}_{t} \lambda_{t, t+1}+r_{t}^{s}-\mathbb{E}_{t} \pi_{t+1} \label{eq:b-3} \tag{B.3}\\
    \lambda_{b, t-1, t}=-\sigma\left(c_{b, t}-c_{b, t-1}\right) \label{eq:b-4} \tag{B.4}\\
    r_{t}^{b}=\frac{\kappa}{R^{b}} q_{t}-q_{t-1} \label{eq:b-5} \tag{B.5}\\
    0=\mathbb{E}_{t} \lambda_{b, t, t+1}+\mathbb{E}_{t} r_{t+1}^{b}-\mathbb{E}_{t} \pi_{t+1} \label{eq:b-6} \tag{B.6}\\
    q_{t}+\widehat{b}_{t}^{F I}=\theta_{t} \label{eq:b-7} \tag{B.7}\\
    {\left[Q b^{F I}(1-\kappa)\right] q_{t}+Q b^{F I} \widehat{b}_{t}^{F I}-\kappa Q b^{F I} \widehat{b}_{t-1}^{F I}+\kappa Q b^{F I} \pi_{t}+r e \cdot \widehat{r e}{ }_{t}=s \cdot \widehat{s}_{t}} \label{eq:b-8} \tag{B.8}\\
    \mathbb{E}_{t} \lambda_{t, t+1}-\mathbb{E}_{t} \pi_{t+1}+\frac{R^{b}}{s p} \mathbb{E}_{t} r_{t+1}^{b}-\frac{R^{s}}{s p} r_{t}^{s}=\omega_{t} \label{eq:b-9} \tag{B.9}\\
    r_{t}^{r e}=r_{t}^{s} \label{eq:b-10} \tag{B.10}\\
    \widehat{p}_{*, t}=\widehat{x}_{1, t}-\widehat{x}_{2, t} \label{eq:b-11} \tag{B.11}\\
    \widehat{x}_{1, t}=(1-\phi \beta) \widehat{p}_{m, t}+(1-\phi \beta) y_{t}+\phi \beta \mathbb{E}_{t} \lambda_{t, t+1}+\epsilon \phi \beta \mathbb{E}_{t} \pi_{t+1}+\phi \beta \mathbb{E}_{t} \widehat{x}_{1, t+1} \label{eq:b-12} \tag{B.12}\\
    \widehat{x}_{2, t}=(1-\phi \beta) y_{t}+\phi \beta \mathbb{E}_{t} \lambda_{t, t+1}+(\epsilon-1) \phi \beta \mathbb{E}_{t} \pi_{t+1}+\phi \beta \mathbb{E}_{t} \widehat{x}_{2, t+1} \label{eq:b-13} \tag{B.13}\\
    \widehat{w}_{t}=\widehat{p}_{m, t}+a_{t} \label{eq:b-14} \tag{B.14}\\
    (1-z) c_{t}+z c_{b, t}=y_{t} \label{eq:b-15} \tag{B.15}\\
    \widehat{v}_{t}^{p}+y_{t}=a_{t}+l_{t} \label{eq:b-16} \tag{B.16}\\
    \widehat{v}_{t}^{p}=0 \label{eq:b-17} \tag{B.17}\\
    \pi_{t}=\frac{1-\phi}{\phi} \widehat{p}_{*, t} \label{eq:b-18} \tag{B.18}\\
    q_{t}+\widehat{b}_{t}^{c b}=\widehat{r e}_{t} \label{eq:b-19} \tag{B.19}\\
    \widehat{b}_{t}=\frac{b^{F I}}{b} \widehat{b}_{t}^{F I}+\frac{b^{c b}}{b} \widehat{b}_{t}^{c b} \label{eq:b-20} \tag{B.20}\\
    c_{b, t}=q_{t}+\widehat{b}_{t} \label{eq:b-21} \tag{B.21}\\
    q e_{t}=\rho_{q} q e_{t-1}+s_{q} \varepsilon_{q, t} \label{eq:b-22} \tag{B.22}\\
    a_{t}=\rho_{A} a_{t-1}+s_{A} \varepsilon_{A, t} \label{eq:b-23} \tag{B.23}\\
    \theta_{t}=\rho_{\theta} \theta_{t-1}+s_{\theta} \varepsilon_{\theta, t} \label{eq:b-24} \tag{B.24}\\
    r_{t}^{r e}=\rho_{r} r_{t-1}^{r e}+\left(1-\rho_{r}\right)\left[\phi_{\pi} \pi_{t}+\phi_{x} x_{t}\right]+s_{r} \varepsilon_{r, t} \label{eq:b-25} \tag{B.25}\\
    q e_{t}=\widehat{r e}_{t} \label{eq:b-26} \tag{B.26}\\
    x_{t}=y_{t}-y_{t}^{*} \label{eq:b-27} \tag{B.27}
\end{gather}
Usamos los mismos números de ecuación que los autores para facilitar la referencia.
\end{document}