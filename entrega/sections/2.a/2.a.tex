\documentclass[../../entrega.tex]{subfiles}
\begin{document}
\subsection{Inciso (a)}
% Problem 2.a: Resuelva el problema de optimización de cada agente y obtenga todas las ecuaciones correspondientes al modelo no lineal.
\subsubsection{Problema del hogar padre}
Comenzamos con el problema del hogar padre, que maximiza su utilidad esperada descontada sujeto a una restricción presupuestaria.
El problema está dado por:
\begin{equation*}
    \max \quad \mathbb{E}_t \sum_{j=0}^{\infty} \beta^j \left[\frac{C_{t + j}^{1-\sigma}-1}{1-\sigma} - \psi \frac{L_{t + j}^{1+\chi}}{1+\chi}\right],
\end{equation*}
sujeto a
\begin{equation*}
    P_t C_t + S_t \leq W_t L_t + R_{t-1}^s S_{t-1} + P_t D_t + P_t D_t^{FI} + P_t T_t - P_t X_t^b - P_t X_t^{FI},
\end{equation*}
donde $C_t$ es consumo, $L_t$ trabajo, $S_t$ bonos de corto plazo, $W_t$ salario nominal, $R_t^s$ tasa de interés nominal de corto plazo, $D_t$ dividendos de las firmas, $D_t^{FI}$ dividendos de los intermediarios financieros, $T_t$ transferencias del gobierno, $X_t^b$ transferencias al hijo y $X_t^{FI}$ transferencias a los intermediarios.
Para resolver este problema, definimos el lagrangiano como
\begin{align*}
    \mathcal{L} & = \mathbb{E}_t \sum_{j=0}^{\infty} \beta^t \left[\frac{C_{t + j}^{1-\sigma}-1}{1-\sigma} - \psi \frac{L_{t + j}^{1+\chi}}{1+\chi}\right]                                                              \\
                & - \mathbb{E}_t \sum_{j=0}^{\infty} \beta^t \varphi_t\left[P_{t + j} C_{t + j} + S_{t + j} - W_{t + j} L_{t + j} - R_{t + j -1}^s S_{t + j -1} - P_{t + j} D_{t + j} - P_{t + j} D_{t + j}^{FI}\right. \\
                & \left.- P_{t + j} T_{t + j} + P_{t + j} X_{t + j}^b + P_{t + j} X_{t + j}^{FI}\right],
\end{align*}
donde $\varphi_t$ es el multiplicador de Lagrange asociado a la restricción presupuestaria.
% \footnote{Podramos escribir la esperanza en $t$ y sumar sobre $j$ con subíndices temporales $t + j$, como hicimos más arriba, pero aprovechamos que el problema es de horizonte infinito y lo podemos expresar de esta forma que es más simple para escribir cuando derivamos.}
Las condiciones de primer orden son:
\begin{align*}
    \{C_t\} & : \beta^t C_t^{-\sigma} = \beta^t \varphi_t P_t            \\
    \{L_t\} & : -\beta^t \psi L_t^\chi = -\beta^t \varphi_t W_t          \\
    \{S_t\} & : -\beta^t \varphi_t + \beta^{t+1} \varphi_{t+1} R_t^s = 0
\end{align*}
De la primera condición, obtenemos $\varphi_t = C_t^{-\sigma}/P_t$.
De la segunda, usando $w_t = W_t/P_t$, obtenemos la ecuación (5):
\begin{equation}
    \psi L_t^\chi = C_t^{-\sigma} w_t,
\end{equation}
que representa la condición intratemporal entre consumo y trabajo.
Definimos el factor de descuento estocástico como:
\begin{equation}
    \Lambda_{t-1,t} = \beta\left(\frac{C_t}{C_{t-1}}\right)^{-\sigma},
\end{equation}
que corresponde a la ecuación (6).
Finalmente, en la tercera condición de primer orden (para $S_t$) reemplazamos $\varphi_t = C_t^{-\sigma}/P_t$ para obtener:
\begin{equation*}
    -\beta^t \frac{C_t^{-\sigma}}{P_t} + \beta^{t+1} \frac{C_{t+1}^{-\sigma}}{P_{t+1}} R_t^s = 0.
\end{equation*}
Multiplicando por $P_t$ y reordenando:
\begin{equation*}
    \beta^t C_t^{-\sigma} = \beta^{t+1} \frac{P_t}{P_{t+1}} C_{t+1}^{-\sigma} R_t^s.
\end{equation*}
Dividiendo ambos lados por $\beta^t C_t^{-\sigma}$:
\begin{equation*}
    1 = \beta \left(\frac{C_{t+1}}{C_t}\right)^{-\sigma} \frac{P_t}{P_{t+1}} R_t^s.
\end{equation*}
Usando la definición de inflación bruta $\Pi_{t+1} = P_{t+1}/P_t$ y notando que el término $\beta(C_{t+1}/C_t)^{-\sigma}$ es el factor de descuento estocástico $\Lambda_{t,t+1}$ definido en la ecuación (6), obtenemos la ecuación de Euler (7)
\begin{equation}
    1 = R_t^s \mathbb{E}_t \Lambda_{t,t+1} \Pi_{t+1}^{-1},
\end{equation}
que representa la condición de optimalidad intertemporal para bonos de corto plazo.

\subsubsection{Rendimiento del bono \emph{long}}
Los bonos de largo plazo en el modelo son perpetuidades que pagan cupones que decaen geométricamente a tasa $\kappa \in [0,1]$.
Un bono emitido en $t$ paga un dólar nominal en $t+1$, $\kappa$ dólares en $t+2$, $\kappa^2$ dólares en $t+3$, y así sucesivamente.
Como se demuestra en \texttt{Bonds.pdf}, no arbitraje implica que el precio en $t$ de un bono emitido en $t-k$ satisface $Q_{t,t-k} = \kappa^k Q_t$, donde $Q_t$ es el precio de un bono nuevo.

El retorno bruto nominal del bono largo entre $t-1$ y $t$ está dado por
\begin{equation}
    R_t^b = \frac{1 + \kappa Q_t}{Q_{t-1}},
\end{equation}
capturando en el numerador la suma del cupón corriente y el valor presente de los cupones futuros.
% En estado estacionario, la duración del bono es $\frac{1}{1-\beta\kappa}$ cuando $\kappa < 1/\beta$, propiedad que se demuestra en \texttt{Bonds.pdf}.
% TODO: maybe remove some of this detail?

\subsubsection{Problema del hogar hijo}
El hogar hijo maximiza una función de utilidad similar a la del padre, pero con dos diferencias fundamentales: no trabaja y descuenta el futuro con un factor $\beta_b < \beta$ (es más impaciente que el padre).
Su problema está dado por
\begin{equation*}
    \max \quad V_{b,t} = \mathbb{E}_t \sum_{j=0}^{\infty} \beta_b^j \left[\frac{C_{b,t+j}^{1-\sigma}-1}{1-\sigma}\right],
\end{equation*}
sujeto a la restricción presupuestaria
\begin{equation*}
    P_t C_{b,t} + B_{t-1} \leq Q_t(B_t - \kappa B_{t-1}) + P_t X_t^b,
\end{equation*}
donde $C_{b,t}$ es su consumo, $B_t$ es la cantidad de bonos largos que emite, $Q_t$ es el precio de dichos bonos, y $X_t^b$ es la transferencia que recibe del padre.

Para resolver este problema, definimos el lagrangiano como
\begin{align*}
    \mathcal{L} & = \mathbb{E}_t \sum_{j=0}^{\infty} \beta_b^j \left[\frac{C_{b,t+j}^{1-\sigma}-1}{1-\sigma}\right]                                                                  \\
                & - \mathbb{E}_t \sum_{j=0}^{\infty} \beta_b^j \varphi_{b,t+j} \left[P_{t+j} C_{b,t+j} + B_{t+j-1} - Q_{t+j}(B_{t+j} - \kappa B_{t+j-1}) - P_{t+j} X_{t+j}^b\right],
\end{align*}
donde $\varphi_{b,t}$ es el multiplicador de Lagrange asociado a la restricción presupuestaria del hijo.

Las condiciones de primer orden son:
\begin{align*}
    \{C_{b,t}\} & : \beta_b^t C_{b,t}^{-\sigma} = \beta_b^t \varphi_{b,t} P_t                            \\
    \{B_t\}     & : -\beta_b^t \varphi_{b,t} Q_t + \beta_b^{t+1} \varphi_{b,t+1}(1 + \kappa Q_{t+1}) = 0
\end{align*}

De la primera condición obtenemos $\varphi_{b,t} = C_{b,t}^{-\sigma}/P_t$.
Definimos el factor de descuento estocástico del hijo como:
\begin{equation}
    \Lambda_{b,t-1,t} = \beta_b\left(\frac{C_{b,t}}{C_{b,t-1}}\right)^{-\sigma},
\end{equation}
que corresponde a la ecuación (13).

De la segunda condición, reemplazando la expresión para $\varphi_{b,t}$ y $\varphi_{b,t+1}$, obtenemos:
\begin{equation*}
    -\frac{C_{b,t}^{-\sigma}}{P_t} Q_t + \beta_b \frac{C_{b,t+1}^{-\sigma}}{P_{t+1}}(1 + \kappa Q_{t+1}) = 0.
\end{equation*}
Multiplicando por $P_t/C_{b,t}^{-\sigma}$:
\begin{equation*}
    -Q_t + \beta_b \frac{P_t}{P_{t+1}} \left(\frac{C_{b,t+1}}{C_{b,t}}\right)^{-\sigma} (1 + \kappa Q_{t+1}) = 0.
\end{equation*}
Reordenando:
\begin{equation*}
    Q_t = \beta_b \frac{P_t}{P_{t+1}} \left(\frac{C_{b,t+1}}{C_{b,t}}\right)^{-\sigma} (1 + \kappa Q_{t+1}).
\end{equation*}
Usando la definición del retorno bruto del bono largo $R_{t+1}^b = (1 + \kappa Q_{t+1})/Q_t$ y la inflación bruta $\Pi_{t+1} = P_{t+1}/P_t$:
\begin{equation*}
    1 = \beta_b \left(\frac{C_{b,t+1}}{C_{b,t}}\right)^{-\sigma} R_{t+1}^b \Pi_{t+1}^{-1}.
\end{equation*}
Tomando esperanza y usando la definición del factor de descuento estocástico del hijo, obtenemos la ecuación (14):
\begin{equation}
    1 = \mathbb{E}_t \Lambda_{b,t,t+1} R_{t+1}^b \Pi_{t+1}^{-1},
\end{equation}
que representa la condición de optimalidad intertemporal para bonos de largo plazo.
\subsubsection{Problema del intermediario financiero}
Los intermediarios financieros viven un período y reciben capital del hogar padre en forma de transferencia $P_t X_t^{FI}$.
Esta transferencia tiene dos componentes: capital nuevo, fijo en $\bar{X}^{FI}$, y el valor de los bonos largos mantenidos por intermediarios previos, valorados a $\kappa Q_t$:
\begin{equation*}
    P_t X_t^{FI} = P_t \bar{X}^{FI} + \kappa Q_t B_{t-1}^{FI}.
\end{equation*}
El intermediario puede atraer depósitos $S_t^{FI}$ del hogar padre, mantener bonos largos $B_t^{FI}$ o reservas $RE_t^{FI}$ en el banco central.
Su balance satisface:
\begin{equation}
    Q_t B_t^{FI} + RE_t^{FI} = S_t^{FI} + P_t X_t^{FI}.\label{eq:eq16}
\end{equation}
El intermediario está sujeto a una ``restricción de apalancamiento'' (\emph{risk-weighted leverage constraint}), definida por
\begin{equation}
    Q_t B_t^{FI} \leq \Theta_t P_t \bar{X}^{FI},\label{eq:eq18}
\end{equation}
donde $\Theta_t$ es un shock exógeno a las condiciones crediticias.
Esta ecuación, siguiendo la interpretación de los autores, dice que el valor de bonos largos en tenencia del intermediario no puede exceder un múltiplo (que varía en el tiempo) de la transferencia del hogar padre.
Se asume que el proceso estocástico que sigue $\Theta_t$ es conocido, y los cambios en $\Theta_t$ se denominan ``shocks al crédito'' (\emph{credit shocks}).
Los bonos largos tienen ponderación de riesgo unitaria, mientras que las reservas tienen ponderación nula.
El intermediario maximiza el valor esperado de sus dividendos futuros, descontados por el factor de descuento estocástico del hogar padre:
\begin{equation*}
    \mathbb{E}_t \Lambda_{t,t+1} \Pi_{t+1}^{-1}\left[(R_{t+1}^b - R_t^s)Q_t B_t^{FI} + (R_t^{re} - R_t^s)RE_t^{FI} + R_t^s P_t X_t^{FI}\right],
\end{equation*}
donde $R_t^s$ es el costo de los depósitos, $R_t^{re}$ el retorno sobre reservas, y $R_{t+1}^b$ el retorno sobre bonos largos.
Definiendo el lagrangiano como
\begin{equation*}
    \begin{aligned}
        \mathcal{L} = \mathbb{E}_t \Lambda_{t,t+1} \Pi_{t+1}^{-1} & \left[(R_{t+1}^b - R_t^s)Q_t B_t^{FI} + (R_t^{re} - R_t^s)RE_t^{FI} + R_t^s P_t X_t^{FI}\right], \\
                                                                  & - \Omega_t[Q_t B_t^{FI} - \Theta_t P_t \bar{X}^{FI}],
    \end{aligned}
\end{equation*}
las condiciones de primer orden son
\begin{align}
    \{B_t^{FI}\}:  & \quad \mathbb{E}_t \Lambda_{t,t+1} \Pi_{t+1}^{-1}(R_{t+1}^b - R_t^s) = \Omega_t, \\
    \{RE_t^{FI}\}: & \quad \mathbb{E}_t \Lambda_{t,t+1} \Pi_{t+1}^{-1}(R_t^{re} - R_t^s) = 0.
\end{align}
Estas son las ecuaciones (19) y (20) del \emph{paper}.
La segunda ecuación implica que el intermediario mantendrá una cantidad indeterminada de reservas cuando $R_t^{re} = R_t^s$.
La primera ecuación muestra que la restricción de apalancamiento vinculante ($\Omega_t > 0$) genera retornos en exceso (\emph{excess return}) sobre los bonos largos.

Para expresar las ecuaciones del intermediario financiero en términos reales, definimos las variables reales de la manera típica, como el cociente entre la variable nominal y el nivel de precios: $b_t^{FI} = B_t^{FI}/P_t$, $re_t = RE_t/P_t$, $s_t = S_t/P_t$ y $X_t^{FI}$ ya está expresado en términos reales.
Dividiendo la ecuación del balance del intermediario (\ref{eq:eq16}) por $P_t$, obtenemos
\begin{equation}
    Q_t b_t^{FI} + re_t = s_t + X_t^{FI},
\end{equation}
que corresponde a la ecuación (A.6) del apéndice.
De manera similar, dividiendo la restricción de apalancamiento (\ref{eq:eq18}) por $P_t$, obtenemos
\begin{equation}
    Q_t b_t^{FI} \leq \Theta_t \bar{X}^{FI},
\end{equation}
que corresponde a la ecuación (A.7) del apéndice.

\subsubsection{Producción: firma mayorista}
La firma mayorista (\emph{wholesale}) opera una tecnología lineal en trabajo para producir el bien mayorista $Y_{m,t}$:
\begin{equation}
    Y_{m,t} = A_t L_t,
\end{equation}
donde $A_t$ es un shock exógeno de productividad que sigue un proceso estocástico conocido, y $L_t$ es el trabajo provisto por el hogar padre.
La firma toma como dados el salario nominal $W_t$ y el precio nominal del bien mayorista $P_{m,t}$, y maximiza sus beneficios, dados por
\begin{equation*}
    \max_{L_t} P_{m,t}A_t L_t - W_t L_t.
\end{equation*}
La condición de primer orden con respecto a $L_t$ es
\begin{equation*}
    P_{m,t}A_t = W_t,
\end{equation*}
que es la típica para firmas con rendimientos constantes a escala.
Dividiendo ambos lados por $P_t$ y definiendo el costo marginal real como $p_{m,t} = P_{m,t}/P_t$ y el salario real como $w_t = W_t/P_t$, obtenemos la ecuación (27) del paper:
\begin{equation}
    w_t = p_{m,t}A_t.
\end{equation}

\subsubsection{Producción: firmas minoristas}
% EMPTY FOR NOW

\subsubsection{Vaciamiento de mercados y producción agregada}
\textcolor{red}{TODO: Review this section}
Las condiciones de vaciamiento de mercado requieren que:
\begin{align*}
    RE_t & = RE_t^{FI}            & \text{(el intermediario mantiene todas las reservas)}             \\
    S_t  & = S_t^{FI}             & \text{(el intermediario mantiene todos los bonos de corto plazo)} \\
    B_t  & = B_t^{FI} + B_t^{c b} & \text{(los bonos largos son mantenidos por el FI o el BC)}
\end{align*}

Para derivar la restricción agregada de recursos, partimos de las restricciones presupuestarias individuales. La restricción del hogar padre es:
\begin{equation*}
    P_t C_t + S_t = W_t L_t + R_{t-1}^s S_{t-1} + P_t D_t + P_t D_t^{FI} + P_t T_t - P_t X_t^b - P_t X_t^{FI},
\end{equation*}
mientras que la del hijo satisface:
\begin{equation*}
    P_t C_{b,t} + B_{t-1} = Q_t(B_t - \kappa B_{t-1}) + P_t X_t^b.
\end{equation*}

Al sumar estas ecuaciones, varios términos se cancelan. La transferencia $X_t^b$ aparece con signo opuesto en ambas ecuaciones, la transferencia $X_t^{FI}$ es un ingreso para el intermediario financiero que afecta sus dividendos $D_t^{FI}$, y las operaciones con bonos y reservas son transferencias entre agentes que se cancelan entre sí.

Por el lado de la producción, los ingresos del sector mayorista son $P_{m,t}Y_{m,t} = W_t L_t$ y los beneficios del sector minorista son $P_t D_t = P_t Y_t - P_{m,t}Y_{m,t}$. Al sumar y simplificar estas expresiones, obtenemos que $P_t Y_t = W_t L_t + P_t D_t$.

Finalmente, sustituyendo estas expresiones y dividiendo por $P_t$, llegamos a la restricción agregada de recursos:
\begin{equation}
    Y_t = C_t + C_{b,t},
\end{equation}
que indica que la producción total debe ser igual a la suma del consumo del padre y del hijo.

Por el lado de la producción, cada minorista $f$ enfrenta una demanda:
\begin{equation*}
    Y_t(f) = \left(\frac{P_t(f)}{P_t}\right)^{-\epsilon} Y_t.
\end{equation*}

Para obtener la producción agregada, integramos sobre todas las variedades:
\begin{equation*}
    Y_t = \int_0^1 Y_t(f)df = \int_0^1 \left(\frac{P_t(f)}{P_t}\right)^{-\epsilon} Y_t df.
\end{equation*}

Dado que cada productor minorista usa una unidad de bien mayorista para producir una unidad de su variedad, la producción total del sector mayorista debe igualar la demanda agregada ajustada por la dispersión de precios:
\begin{equation*}
    Y_{m,t} = \int_0^1 Y_t(f)df = Y_t \int_0^1 \left(\frac{P_t(f)}{P_t}\right)^{-\epsilon} df = Y_t v_t^p,
\end{equation*}
donde $v_t^p$ es la medida de dispersión de precios definida como:
\begin{equation*}
    v_t^p = \int_0^1 \left(\frac{P_t(f)}{P_t}\right)^{-\epsilon} df.
\end{equation*}

Usando la función de producción de las firmas mayoristas $Y_{m,t} = A_t L_t$, obtenemos:
\begin{equation}
    Y_t v_t^p = A_t L_t,
\end{equation}
que corresponde a la ecuación (31) del paper.

Finalmente, las $27$ ecuaciones que caracterizan el modelo no lineal son las siguientes, donde usamos los mismos números de ecuación que los autores para facilitar la referencia.
\begin{gather}
    \psi L_t^\chi = C_t^{-\sigma} w_t \label{eq:2-5} \tag{2.5} \\
    \Lambda_{t-1,t} = \beta\left(\frac{C_t}{C_{t-1}}\right)^{-\sigma} \label{eq:2-6} \tag{2.6} \\
    1 = R_t^s \mathbb{E}_t \Lambda_{t,t+1} \Pi_{t+1}^{-1} \label{eq:2-7} \tag{2.7} \\
    \Lambda_{b,t-1,t} = \beta_b\left(\frac{C_{b,t}}{C_{b,t-1}}\right)^{-\sigma} \label{eq:2-13} \tag{2.13} \\
    1 = \mathbb{E}_t \Lambda_{b,t,t+1} R_{t+1}^b \Pi_{t+1}^{-1} \label{eq:2-14} \tag{2.14} \\
    \mathbb{E}_t \Lambda_{t,t+1} \Pi_{t+1}^{-1}(R_{t+1}^b - R_t^s) = \Omega_t \label{eq:2-19} \tag{2.19} \\
    \mathbb{E}_t \Lambda_{t,t+1} \Pi_{t+1}^{-1}(R_t^{re} - R_t^s) = 0 \label{eq:2-20} \tag{2.20} \\
    Q_t b_t^{FI} + re_t = s_t + X_t^{FI} \label{eq:a-6} \tag{A.6} \\
    Q_t b_t^{FI} \leq \Theta_t \bar{X}^{FI} \label{eq:a-7} \tag{A.7} \\
    w_t = p_{m,t}A_t \label{eq:2-27} \tag{2.27} \\
    p_{*, t}=\frac{\epsilon}{\epsilon-1} \frac{x_{1, t}}{x_{2, t}} \label{eq:a-1} \tag{A.1}\\
    x_{1, t}=p_{m, t} Y_{t}+\phi \mathbb{E}_{t} \Lambda_{t, t+1} \Pi_{t+1}^{\epsilon} x_{1, t+1}  \label{eq:a-2}\tag{A.2}\\
    x_{2, t}=Y_{t}+\phi \mathbb{E}_{t} \Lambda_{t, t+1} \Pi_{t+1}^{\epsilon-1} x_{2, t+1} \label{eq:a-3} \tag{A.3}\\
    Y_{t}=C_{t}+C_{b, t} \label{eq:2-30}\tag{2.30}\\
    Y_{t} v_{t}^{p}=A_{t} L_{t} \label{eq:2-31}\tag{2.31}\\
    \ln R_{t}^{s}=\left(1-\rho_{r}\right) \ln R^{s}+\left(1-\rho_{r}\right)\left[\phi_{\pi}\left(\ln \Pi_{t}-\ln \Pi\right)+\phi_{x}\left(\ln Y_{t}-\ln Y_{t}^{*}\right)\right]+s_{r} \varepsilon_{r, t}  \label{eq:a-14}\tag{A.14}\\
    Q_{t} b_{t}^{c b}=r e_{t} \label{eq:a-8}\tag{A.8}\\
    QE_t = Q_t b_t^{cb} \label{eq:a-10}\tag{A.10}\\
    \ln Q E_{t}=\left(1-\rho_{q}\right) \ln Q E+\rho_{q} \ln Q E_{t-1}+s_{q} \varepsilon_{q, t} \label{eq:a-15}\tag{A.15}\\
    C_{b, t}=Q_{t} b_{t}\label{eq:a-11} \tag{A.11}\\
    b_t = b_t^{FI} + b_t^{cb} \label{eq:a-9} \tag{A.9}\\
    1=(1-\phi) p_{*, t}^{1-\epsilon}+\phi \Pi_{t}^{\epsilon-1}\label{eq:a-4} \tag{A.4}\\
    v_{t}^{p}=(1-\phi) p_{*, t}^{-\epsilon}+\phi \Pi_{t}^{\epsilon} v_{t-1}^{p} \label{eq:a-5}\tag{A.5}\\
    \ln X_t = \ln Y_t - \ln Y_t^* \label{eq:a-16} \tag{A.16}\\
    \ln A_{t}=\rho_{A} \ln A_{t-1}+s_{A} \varepsilon_{A, t}\label{eq:a-12}  \tag{A.12}\\
    \ln \Theta_{t}=\left(1-\rho_{\theta}\right) \ln \Theta+\rho_{\theta} \ln \Theta_{t-1}+s_{\theta} \varepsilon_{\theta, t}\label{eq:a-13} \tag{A.13}
\end{gather}

\end{document}