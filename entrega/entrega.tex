%% !TeX root = filename
\documentclass[10pt,a4paper]{article}
\usepackage[a4paper, margin=3cm]{geometry}
\usepackage{subfiles}
\usepackage{graphicx}
\graphicspath{{./figures/}}
\usepackage[utf8]{inputenc}
\usepackage[T1]{fontenc}
\usepackage[spanish]{babel}
\usepackage{booktabs}
\usepackage{dirtree}
% \usepackage[style=authoryear, backend=biber, language=spanish]{biblatex}
\usepackage[authordate-trad,backend=biber,language=spanish]{biblatex-chicago}
\addbibresource{references.bib}
\usepackage{amsmath}
\usepackage{amsfonts}
\usepackage{amssymb}
%\usepackage[extreme]{savetrees}
\usepackage{caption} 
\usepackage{enumerate}
\usepackage{mdframed}
\usepackage{csquotes}
\renewcommand{\mkbegdispquote}[2]{\itshape}
\usepackage{xfrac}
\usepackage{tikz}
\usepackage{pgfplots}
\usepackage{placeins}
\usepackage{subcaption}
\usepackage[hidelinks]{hyperref}
\usepackage[useregional]{datetime2}
% \setcounter{tocdepth}{2}
\setcounter{secnumdepth}{1}
\pgfplotsset{compat=1.5}
% \setlength\parindent{0pt}
\author{\href{mailto:felipetappata@gmail.com}{Felipe Tappata}}
\title{Examen Domiciliario}
%\date{\DTMdisplaydate{2021}{09}{01}{-1}}
\renewcommand{\l}[2]{\href{#2}{#1}}
% \captionsetup[table]{skip=10pt}
\makeatletter
\def\@maketitle{%
	\newpage
	\null
	\vskip 2em%
	\begin{center}%
		\let \footnote \thanks
		{\LARGE \@title \par}%
		\vskip 1em%
		{Macroeconomía II\par}% Write subtitle here
		\vskip 1.5em%
		{\large
			\lineskip .5em%
			\begin{tabular}[t]{c}%
				\@author
			\end{tabular}\par}%
		\vskip 1em%
		{\large \@date}%
	\end{center}%
	\par
	\vskip 1.5em}
\makeatother
\begin{document}
\maketitle

\noindent\textit{Este trabajo fue entregado el domingo 22 de diciembre, según lo acordado con el profesor para alumnos afectados por la extensión del plazo.}
\section{Sobre el trabajo}
Este trabajo consiste en una resolución del examen domiciliario final de \emph{Macroeconomía II} de la Maestría en Economía de la Universidad Torcuato Di Tella, cuyas consignas han sido reproducidas en sus correspondientes secciones para facilitar la referencia.
El examen consta de dos ejercicios, cada uno con varios incisos, de los cuales algunos implican el uso de código para replicar resultados.
La estructura del subdirectorio entregado es la siguiente.
\dirtree{%
	.1 entrega/.
	.2 entrega.pdf.
	.2 README.md.
	.2 code/.
	.3 swz/.
	.3 werning/.
	.3 dynare/.
	.3 uhlig toolkit/.
}
Los nombres de los archivos y directorios son relativamente sugestivos de sus contenidos, pero el Apéndice \textcolor{red}{TODO: agregar referencia} contiene una descripción más detallada.
\section[Some Unpleasant Monetarist Arithmetic]{\emph{Some Unpleasant Monetarist Arithmetic}}
\subsection{Consigna}
\begin{mdframed}
	\subsubsection[Ejercicio 1]{1. [35 puntos] Werning (2024)}
	Lea el siguiente artículo:
	Werning, Iván (2024). "Recalculating Sargent and Wallace's 'Some Unpleasant Monetarist Arithmetic'," Federal Reserve Bank of Minneapolis Quarterly Review, Vol. 44, No. 3 (November), pp. 1-19.
	\begin{enumerate}[(a)]
		\item Derive detalladamente, e interprete, todos los resultados del artículo (excepto los del Apéndice B).
		\item Replique las figuras 1 y 2 . La demanda de dinero utilizada es
		      \begin{equation*}
			      L\left(\pi_{t, t+1}\right)=\frac{\gamma_{1}}{2}-\frac{\gamma_{2}}{2}\left(1+\pi_{t, t+1}\right),
		      \end{equation*}
		      con $\gamma_{1}=3$ y $\gamma_{2}=2.5$. La tasa de interés real es $r=0.05$. Para la Figura 2 utilice $\frac{1+\omega_{0}}{1+\Delta_{0}}=\frac{M_{-1,0}+\left(1+i_{-1,0}\right) B_{-1,0}}{M_{0,1}}=0.243$ y $D=0.336$.
	\end{enumerate}
\end{mdframed}

\subfile{sections/1.a/1.a.tex}
\subfile{sections/1.b/1.b.tex}

\section[The Four-Equation New Keynesian Model]{\emph{The Four-Equation New Keynesian Model}}
\subsection{Consigna}
\begin{mdframed}
	\subsubsection[Ejercicio 2]{2. [65 puntos] Sims, Wu, and Zhang (2023)}
	En este ejercicio nos vamos a concentrar en las secciones I y II del siguiente artículo:\\
	Sims, E., Wu J., and Zhang, J. (2023). "The Four-Equation New Keynesian Model," The Review of Economics and Statistics, Vol. 105, Issue 4 (July), pp. 931-947.
	\begin{enumerate}[(a)]
		\item Resuelva el problema de optimización de cada agente y obtenga todas las ecuaciones correspondientes al modelo no lineal.\footnote{Puede consultar el apéndice \emph{online} del artículo.}\footnote{Para complementar la descripción de los bonos de largo plazo, puede leer la nota adjunta \texttt{Bonos.pdf}.}
		\item Log-linealice todas las ecuaciones del modelo.\footnote{Puede consultar el apéndice \emph{online} del artículo.}
		\item Demuestre que el sistema de ecuaciones log-lineales se puede reducir al siguiente Modelo Neokeynesiano de Cuatro Ecuaciones:
		      \begin{equation*}
			      \begin{aligned}
				       & x_{t}=\mathbb{E}_{t} x_{t+1}-\frac{1-z}{\sigma}\left(r_{t}^{s}-\mathbb{E}_{t} \pi_{t+1}-r_{t}^{*}\right)-z\left(\bar{b}^{F I}\left(\mathbb{E}_{t} \theta_{t+1}-\theta_{t}\right)+\bar{b}^{c b}\left(\mathbb{E}_{t} q e_{t+1}-q e_{t}\right)\right) \\
				       & \pi_{t}=\gamma \zeta x_{t}-\frac{z \gamma \sigma}{1-z}\left(\bar{b}^{F I} \theta_{t}+\bar{b}^{c b} q e_{t}\right)+\beta \mathbb{E}_{t} \pi_{t+1}                                                                                                   \\
				       & r_{t}^{s}=\rho_{r} r_{t-1}^{s}+\left(1-\rho_{r}\right)\left(\phi_{\pi} \pi_{t}+\phi_{x} x_{t}\right)+v_{t}                                                                                                                                         \\
				       & q e_{t}=\rho_{q} q e_{t-1}+\epsilon_{q, t}
			      \end{aligned}
		      \end{equation*}
		      con
		      \begin{equation*}
			      \begin{aligned}
				       & r_{t}^{*}=-\frac{\left(1-\rho_{a}\right)(1+\chi) \sigma}{\sigma+(1-z) \chi} a_{t} \\
				       & y_{t}^{*}=\frac{(1-z)(1+\chi)}{\sigma+(1-z) \chi} a_{t}                           \\
				       & y_{t}=y_{t}^{*}+x_{t}                                                             \\
				       & a_{t}=\rho_{a} a_{t-1}+\epsilon_{a, t}                                            \\
				       & \theta_{t}=\rho_{\theta} \theta_{t-1}+\epsilon_{\theta, t}                        \\
				       & v_{t}=\rho_{v} v_{t-1}+\epsilon_{r, t}
			      \end{aligned}
		      \end{equation*}
		      donde $\mathbb{E}\left(\epsilon_{q, t}\right)=0, \operatorname{Var}\left(\epsilon_{q, t}\right)=s_{q}^{2}, \mathbb{E}\left(\epsilon_{a, t}\right)=0, \operatorname{Var}\left(\epsilon_{a, t}\right)=s_{a}^{2}, \mathbb{E}\left(\epsilon_{\theta, t}\right)=0, \operatorname{Var}\left(\epsilon_{\theta, t}\right)=s_{\theta}^{2}, \mathbb{E}\left(\epsilon_{r, t}\right)=0 y$ $\operatorname{Var}\left(\epsilon_{r, t}\right)=s_{r}^{2}$. Los autores suponen que $\rho_{v}=0$.
		\item Describa e interprete cada una de las ecuaciones del modelo. Demuestre que la ecuación (36) del artículo se puede deducir del sistema anterior. Explique bajo qué condiciones el sistema anterior se reduce al modelo neokeynesiano estándar de tres ecuaciones. Liste cada una de las ecuaciones del modelo de tres ecuaciones.
		\item Escriba las ecuaciones del modelo en forma matricial, utilizando el formato \emph{brute force} de Uhlig. Escriba un código para el \emph{toolkit} de Uhlig y resuelva el modelo utilizando la siguiente calibración (muy similar a la de la Tabla 1 del artículo): $\beta=0.995, z=0.33, \sigma=\chi=1, \bar{b}^{F I}=0.697, \bar{b}^{c b}=0.303, \gamma=0.0846$, $\zeta=2.4925, \rho_{r}=\rho_{a}=\rho_{\theta}=\rho_{q}=0.8, \rho_{v}=0, \phi_{\pi}=1.5, \phi_{x}=0$. Los desvíos estándar de los shocks no son relevantes porque nos vamos a concentrar en las funciones de impulso-respuesta, pero los autores utilizan los siguientes valores: $s_{r}=s_{q}=s_{\theta}=0.01$, y $s_{a}=0.0125$. Presente la solución del modelo (matrices $P$ y $Q)$. Reporte las funciones de impulso-respuesta para cada shock.
		\item Escriba un código en \emph{Dynare} y demuestre que obtiene la misma solución del inciso anterior.
		\item Replique la Figura 1 del artículo (los tres paneles).\footnote{Note que para hacerlo tendrá que resolver también el modelo de tres ecuaciones}
		      Preste atención a la manera en que los autores normalizan los shocks. En el panel (a), la función de impulso-respuesta del producto está calculada para un shock de tamaño $\epsilon_{a, 0}=s_{a}=0.0125$. Comente las similitudes y diferencias entre los resultados de los dos modelos.
		\item Obtenga la ecuación (39) del artículo. Replique la Figura 2.
	\end{enumerate}
\end{mdframed}

\subfile{sections/2.a/2.a.tex}
\subfile{sections/2.b/2.b.tex}
% \subfile{sections/2.c/2.c.tex}
% \subfile{sections/2.d/2.d.tex}
% \subfile{sections/2.e/2.e.tex}
% \subfile{sections/2.f/2.f.tex}
\subfile{sections/2.g/2.g.tex}
\subfile{sections/2.h/2.h.tex}

\appendix
\section{Descripción detallada de archivos}
\label{sec:appendix}
\textcolor{red}{TODO: completar}

\printbibliography
\end{document}